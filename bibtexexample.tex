\documentclass[11pt]{article}%{amsart} %{article}
\usepackage{geometry}
\geometry{letterpaper, tmargin=1in,bmargin=1in,lmargin=1in,rmargin=1in}

\begin{document}
\title{A Bibliography Example}
\author{Jerome Fung}
\date{July 16, 2013}
\maketitle

\section{Mie Scattering}
The scattering of light by a homogenous sphere is highly relevant to
colloid science, meteorology, and astronomy, to name but a few
fields. The exact solution of Maxwell's equations for this scattering
problem is known as the Mie solution. In the modern formulation given
by Bohren \& Huffman \cite{Bohren83}, finding the scattered
electromagnetic fields consists of solving for the vector
eigenfunctions of a sphere (known as vector spherical harmonics),
expanding an incident plane wave in terms of vector spherical
harmonics, and then applying appropriate boundary conditions at the
surface of the sphere. The necessary electromagnetic theory is
discussed in Jackson \cite{Jackson99}. Algorithms still in use today
for the non-trivial task of numerically computing results from the
analytic solution are clearly discussed by Wiscombe \cite{Wiscombe80}.  


\bibliographystyle{plain}
\bibliography{examplebibfile}

\end{document}



